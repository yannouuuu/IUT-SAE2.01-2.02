\documentclass[fontsize=10pt,oneside]{scrreprt}
\usepackage{ulille-rapport}
\usepackage{lipsum}

%%%% Choix de la langue du document
\selectlanguage{french}
%\selectlanguage{english}

%%%% Choix des couleurs du document
%% (la commande \ListeCouleurs génère une table des couleurs disponibles)
%% /!\ Les modifications de couleurs ci-dessous ne devraient être utilisées
%% que dans le préambule et être valables pour tout le document.
% \colorlet{ULilleFond}{Blanc}                % fond de page
% \colorlet{ULilleCrayon}{Noir}               % texte courant
% \colorlet{ULilleSurligneur}{JauneDetermine} % fond de la commande \surligne{...}
% \colorlet{ULilleStructure}{Noir}            % couleur des éléments de structure
% \colorlet{ULilleSecondaire}{RougeAction}    % couleur de texte utilisée par
                                              % la commande \colorie{...}
% \colorlet{ULilleLiens}{BleuHorizon}         % couleur des liens URL

%%%% Titre, auteur, pied de page
%% Le logo en haut à gauche de la page est chargé par les commandes
%% \maketitle ou \logoULille
%% Il peut être adapté en redéfinissant la commande \includeLogo comme suit :
% \renewcommand{\includeLogo}{\includegraphics[height=2cm]{Logo.sans.baseline-Horizontal-RVB-Noir.pdf}}
\subject{Rapport}
\title{Exemple d'utilisation du style ULille}
\author{Pierre Boulet}
\direction{VP Numérique}

\begin{document}
\maketitle

\pagecolor{Blanc}

\tableofcontents

\part*{Partie/intercalaire}

\chapter{Chapitre}

\section{Gestion des couleurs}

Liste des couleurs disponibles (voir le site \url{https://identite.univ-lille.fr}) :

\ListeCouleurs

Mise en valeur par la couleur : \surligne{texte surligné}, \colorie{texte colorié}.
\emph{La mise en valeur par l'italique reste disponible.}

\section{Polices}
La police de texte par défaut est la Marianne. La police complémentaire \textsf{Spectral} est disponible par la commande \verb|\textsf|. Et le texte à chasse fixe est composé en \texttt{Droid Sans Mono}.

\section{Section}
\lipsum[1-2]


\section{Section}
\subsection{Sous-section}
\lipsum[3]
\subsection{Sous-section}
\lipsum[4]

\section*{Section non numérotée}
Exemple de liste :
\begin{itemize}
  \item item
  \begin{itemize}
    \item sous item.
  \end{itemize}
\end{itemize}

\end{document}
